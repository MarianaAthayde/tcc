%
% Documento: Resumo (Português)
%

\begin{resumo}

Com o avanço da tecnologia e a participação cada vez mais frequente de robôs na nossa sociedade, os estudos na área da robótica vem ganhando ênfase e vem promovendo e conquistando seu espaço no mundo contemporâneo. Além de fornecer soluções cada vez melhores e mais baratas para indústria, ela vem inovando na interação das pessoas com o resto do mundo. Do ponto de vista da robótica móvel, existe uma grande área de atuação e problemas que podem ser solucionados de forma a facilitar e proteger a vida das pessoas. Este trabalho tem como objetivo o estudo de estratégias de controle de formação, assim como, a implementação dessas estratégias em uma plataforma de relativo baixo custo e didática. As duas estratégias a serem implementadas neste trabalho consistem em sincronismo em paralelo de uma frota de robôs, onde eles devem se alinhar paralelamente e seguir se deslocando em linha reta, como em um problema de varredura em paralelo, e a outra estratégia é fazer com que a frota de robôs localize e rodeie um alvo e se distribua de forma balanceada de acordo com o número de robôs da frota. Neste trabalho não se teve como intuito a implementação de um tratamento de colisão entre os diferentes robôs, nem mesmo se pretendeu interagir com o ambiente e evitar a  colisão com obstáculos externos. Para cumprir com os objetivos deste trabalho, foi projetado um sistema de controle em cascata e foram realizados diversos experimentos com diferentes controladores. Os resultados mostram que os \emph{encoders} da própria plataforma utilizados para realizar a odometria do robô são suficientemente precisos para que sejam utilizados no sistema de localização dos robôs e que as estratégias adotadas foram eficientes para que o time de robôs não só se alinhasse paralelamente, como também localizasse e circulasse um alvo reajustando a sua configuração em função do número de robôs.

\textbf{Palavras-chave}: Estratégias de controle de formação. Robótica móvel. Lego Mindstorms. 
\end{resumo}
