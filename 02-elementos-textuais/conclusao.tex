%
% Documento: Conclusão
%

\chapter{Conclusão}
\label{chap:conclusao}

Espera-se que o uso do estilo de formatação LATEX adequado às Normas para Elaboração de Trabalhos Acadêmicos do CEFET-MG ({\ttfamily abntex2-cefetmg.cls}) facilite a escrita de documentos no âmbito desta instituição e aumente a produtividade de seus autores. Para usuários iniciantes em LATEX, além da bibliografia especializada já citada, existe ainda uma série de recursos \cite{CTAN2009} e fontes de informação \cite{TeX-Br2009,Wikibooks2009} disponíveis na Internet.

Recomenda-se o editor de textos Kile como ferramenta de composição de documentos em LATEX para usuários Linux. Para usuários Windows recomenda-se o editor TEXnicCenter \cite{TeXnicCenter2009}. O LATEX normalmente já faz parte da maioria das distribuições Linux, mas no sistema operacional Windows é necessário instalar o software MiKTeX \cite{MiKTeX2009}.

Além disso, recomenda-se o uso de um gerenciador de referências como o JabRef\index{JabRef} \cite{JabRef2009} ou Mendeley \cite{Mendeley2009} para a catalogação bibliográfica em um arquivo BIBTEX, de forma a facilitar citações através do comando \verb#\cite{}# e outros comandos correlatos do pacote ABNTEX. A lista de referências deste documento foi gerada automaticamente pelo software LATEX + BIBTEX a partir do arquivo {\ttfamily refbase.bib}, que por sua vez foi composto com o gerenciador de referências JabRef.

\section{Trabalhos futuros}
\label{sec:trabalhosFuturos}

Inserir seu texto aqui...
