\chapter{Problema 1}
\label{chap:Simulacoes}

As simulações realizadas neste trabalho, consideram o robô como um elemento pontual, conforme descrito nas equações \ref{eq:posiçãox}, \ref{eq:posiçãoy} e \ref{eq:posiçãotheta} e foram realizadas com intuito de se observar a viabilidade de implementação da estratégia de controle. 

Primeiramente, foi realizado a simulação da primeira estratégia de abordagem do primeiro problema, em que tem-se uma frota de \emph{N} robôs separados entre si no eixo $y$ e com uma posição randômica no eixo $x$, todos dispostos na mesma direção e sentido. A primeira estratégia elimina os problemas de colisão, visto que os robôs começam separados, no mesmo sentido e como são impossibilitados a imprimir uma velocidade angular pela estratégia adotada, seguem em linha reta com velocidade linear negativa ou positiva.  