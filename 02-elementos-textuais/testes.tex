\chapter{Teste das Malhas de Controle}
\label{chap:testesMalhas}
Para alcançar o objetivo final deste trabalho\footnote{Todos os dados dos testes estão disponíveis em: https://github.com/MarianaAthayde/tcc/tree/master/Testes. Os códigos estão no link: https://github.com/MarianaAthayde/tcc/tree/master/Implementações/Implementações - Consolidadas.}, fez-se uma modularização do problema que foi validada e integrada, módulo a módulo. Primeiramente será testada a malha 1 de controle de velocidade angular das rodas. %Ou seja, após dado um comando de velocidade angular para roda quanto tempo a malha levará para estabilizar o sistema, isto é, para que o erro convirja para zero. 
Ou seja, ao passar um comando de velocidade angular para roda o sistema deverá estabilizar, convergindo o erro do mesmo para zero, em um determinado tempo que espera-se ser suficiente para atingir com o objetivo proposto.

O primeiro teste a ser realizado será na malha de controle mais interna com cinco controladores diferentes: P, PI, PID, Incremental e o quinto, através de uma função polinomial que converte a velocidade angular desejada na potência a ser aplicada. Para utilizar essa função polinomial é necessário recorrer a um método da linguagem \emph{NXC} de ativação dos motores que já implementa um controlador de malha fechada. Essa função foi obtida por meio de um mapeamento da resposta do sistema às potências aplicadas. Em seguida foi realizada uma aproximação de 4º grau com os dados obtidos.
%Essa função foi obtida por meio de um mapeamento das potências aplicadas e as velocidades adquiridas pelo motor. Em seguida foi realizada uma aproximação de 4º grau. %Esse mapeamento, assim como, a função aproximada foi realizada pelo Professor Tales Argolo Jesus e será utilizada neste trabalho.

Para que a malha de controle interno cumpra seu objetivo é necessário que o erro entre a velocidade desejada e a velocidade real em cada roda tenda a zero. Ou seja, desejando-se que o robô tenha uma velocidade linear de $0.1 m/s$ e uma velocidade angular de $0.6 rad/s$, faz-se uma conversão desses valores utilizando da \autoref{eq:convVelAngRodas}, encontrando os valores desejados para cada uma das rodas. Feito isso, calcula-se o erro entre a velocidade desejada e a real de cada roda e passa esse erro como entrada para o controlador que deve aplicar uma potência nos motores pretendendo atingir a velocidade desejada.

Para realizar o teste com a malha interna foram aplicadas três entradas distintas e foi observado o tempo de conversão do sistema para um estado estacionário e se o sistema estava convergindo para um erro próximo de zero. Utilizando-se de um intervalo de amostragem de cerca de 25 milissegundos, aplicou-se nas primeiras 100 amostras uma velocidade linear e angular nula. Feito isso nas próximas 400 amostras aplicou-se uma velocidade linear de $0.189 m/s$ e uma velocidade angular nula (velocidade angular de cada roda igual a $8.75 rad/s$) e nas 500 amostras seguintes manteve-se a velocidade linear anterior e foi acrescentada uma velocidade angular de $0.628 rad/s$ (sendo, $10.38 rad/s$ e $7.12 rad/s$, as velocidades angulares das rodas direita e esquerda, respectivamente). É importante\footnote{Com intuito de facilitar a visualização dos tempos de ajuste do sistema, algumas amostras de dados foram retiradas. Essas amostras não interferem no entendimento do comportamento do sistema, visto que foram retiradas faixas em que o sistema já estava estável.}  destacar que no primeiro momento o robô deve permanecer parado, adquirir uma velocidade linear e caminhar em linha reta com uma velocidade de $0.189 m/s$, para então iniciar uma trajetória circular de raio de $30 cm$ e período de $10 s$.

\section{Malha 1: Controlador Incremental}
\label{m1ContInc}

Primeiro a malha interna foi testada com um controlador incremental que consiste basicamente em aumentar ou diminuir a potência das rodas, de acordo com o erro. Se o erro for positivo, incrementa-se positivamente a potência, se for negativo, decrementa-se negativamente a potência. Essa potência será passada como referência para as funções de comando dos motores, que por sua vez, só aceitam valores na faixa entre -100 e 100.% A potência das rodas aceita pela plataforma é uma variável que está na faixa entre -100 e 100. 
Foram testados os seguintes valores como incremento: 1, 3 e 6 e como podemos observar nos gráficos \ref{fig:contInc1},\ref{fig:contInc6} e \ref{fig:contInc3} e na \autoref{t:cinc}, o controlador de incremento igual a um é suave e possui oscilações aceitáveis mas, o tempo para estabilização do sistema é maior que o dobro do tempo obtido com o incremento igual a seis. Entretanto, o controlador com incremento igual a seis é muito agressivo, oscilando com amplitude dobrada se comparado ao controlador de incremento igual a um. 

Fazendo-se o incremento igual a 3 tem-se um custo-benefício interessante pois, o sistema continua razoavelmente suave, com oscilações bem próximas ao do controlador de incremento igual a um e com um tempo de resposta %o dobro de vezes mais rápido.
duas vezes menor. Já, comparado ao de incremento igual a seis, ele apresenta um tempo de resposta bem próximo e com a metade da oscilação, com um controle bem mais suave. Portanto, o incremento igual a 3 parece ser o mais adequado para a malha interna se comparado ao controlador com outros valores de incremento. 

\begin{table}[!htb]
	\centering
	\begin{tabular}{|c|c|c|c|}
		\hline
		\begin{tabular}[c]{@{}c@{}}Características/\\ Incrementos\end{tabular} & \textbf{\begin{tabular}[c]{@{}c@{}}Oscilação\\ ($rad/s$)\end{tabular}} & \textbf{\begin{tabular}[c]{@{}c@{}}Tempo de \\ Assentamento ($s$)\end{tabular}} & \textbf{Controle} \\ \hline
		\textbf{1}                                                             & $\pm 0.6 rad/s$                                                            & $\approx 2s$                                                                      & Suave             \\ \hline
		\textbf{3}                                                             & $\pm 0.6 rad/s$                                                            & $\approx 1s$                                                                      & Suave             \\ \hline
		\textbf{6}                                                             & $\pm 1.2 rad/s$                                                            & $\approx 0.75s$                                                                   & Agressivo         \\ \hline
	\end{tabular}
	\caption{Malha1: Controlador Incremental}
	\label{t:cinc}
\end{table}

\begin{figure}[!htb]
	\centering
	\begin{subfigure}{1.0\textwidth}
		\centering
		\includegraphics[width=.9\linewidth]{./Testes/Malha1/Incremental/Malha1_Inc1}
		\caption{Erro da malha interna com incremento = 1}
		\label{fig:contInc1}
	\end{subfigure}
	\begin{subfigure}{1.0\textwidth}
		\centering
		\includegraphics[width=.9\linewidth]{./Testes/Malha1/Incremental/Malha1_Inc6}
		\caption{Erro da malha interna com incremento = 6}
		\label{fig:contInc6}
	\end{subfigure}
	\begin{subfigure}{1.0\textwidth}
		\centering
		\includegraphics[width=.9\linewidth]{./Testes/Malha1/Incremental/Malha1_Inc3}
		\caption{Erro da malha interna com incremento = 3}
		\label{fig:contInc3}
	\end{subfigure}
	\caption{Experimentos com Controlador Incremental}
	\label{fig:contInc}
\end{figure}

\section{Malha 1: Controladores P, PI e PID}
\label{m1ContPID}

Foram realizados testes com controladores do tipo P, PI e PID, os ganhos foram ajustados de forma empírica e a \autoref{t:cpid}, assim como as figuras \ref{fig:contP} e \ref{fig:contPID} mostram alguns dos resultados  obtidos. Pode-se observar que o melhor ganho obtido a partir dos testes realizados foi o do controlador PID com ganho proporcional igual a $10$, ganho integrativo igual a $15$ e ganho derivativo igual a $0,2$, visto que ele apresentou o menor tempo de resposta com uma oscilação razoável e inferior aos outros ganhos. 

\begin{table}[!htb]
	\centering
	\begin{tabular}{|c|c|c|c|c|}
		\hline
		\textbf{\begin{tabular}[c]{@{}c@{}}Características/\\ Ganhos do controlador\end{tabular}} & \textbf{\begin{tabular}[c]{@{}c@{}}Valor de \\Estabilização\\ ($rad/s$)\end{tabular}} & \textbf{\begin{tabular}[c]{@{}c@{}}Oscilação\\ ($rad/s$)\end{tabular}} & \textbf{\begin{tabular}[c]{@{}c@{}}Tempo de \\ Assentamento ($s$)\end{tabular}} & \textbf{Controle} \\ \hline
		\textbf{P: 10}                                                                            & $3$ $rad/s$                                                               & $1.3 rad/s$                                                            & $0.5s$                                                                    & Suave             \\ \hline
		\textbf{P:10 - I:15}                                                                      & $0$ $rad/s$                                                               & $0.65 rad/s$                                                           & $3.25s$                                                                      & Suave             \\ \hline
		\textbf{P:10 - I:17}                                                                      & $0$ rad/s                                                               & $0.65 rad/s$                                                            & $2.9s$                                                                    & Suave             \\ \hline
		\textbf{P:10 - I:15 - D:0.2}                                                              & $0$ $rad/s$                                                               & $0.6 rad/s$                                                            & $2.9s$                                                                    & Suave             \\ \hline
		\textbf{P:10 - I:15 - D:0.6}                                                              & $0$ $rad/s$                                                               & $1.3 rad/s$                                                           & $2s$                                                                      & Agressivo         \\ \hline
	\end{tabular}
	\caption{Experimentos com Controlador P/PI/PID}
	\label{t:cpid}
\end{table}


\begin{figure}[!htb]
	\centering
	\begin{subfigure}{1.0\textwidth}
		\centering
		\includegraphics[width=.9\linewidth]{./Testes/Malha1/PID/m1_kp10}
		\caption{Controlador Proporcional (kp = 10)}
		\label{fig:contP}
	\end{subfigure}
	\begin{subfigure}{1.0\textwidth}
		\centering
		\includegraphics[width=.9\linewidth]{./Testes/Malha1/PID/m1_kp10ki15}
		\caption{Controlador Proporcional - Integral (kp = 10 e ki = 15)}
		\label{fig:contPI1}
	\end{subfigure}
	\caption{Experimentos com Controlador P e PI}
	\label{fig:contP}
\end{figure}
\begin{figure}[!htb]
	\centering
	\begin{subfigure}{1.0\textwidth}
		\centering
		\includegraphics[width=.9\linewidth]{./Testes/Malha1/PID/m1_kp10ki15kd06}
		\caption{Controlador Proporcional - Integral - Derivativo (kp = 10, ki = 15 e kd = 0.6)}
		\label{fig:contPID2}
	\end{subfigure}
	\begin{subfigure}{1.0\textwidth}
		\centering
		\includegraphics[width=.9\linewidth]{./Testes/Malha1/PID/m1_kp10ki17}
		\caption{Controlador Proporcional - Integral - Derivativo (kp = 10 e ki = 17)}
		\label{fig:contPI2}
	\end{subfigure}
	\begin{subfigure}{1.0\textwidth}
		\centering
		\includegraphics[width=.9\linewidth]{./Testes/Malha1/PID/m1_kp10ki15kd02}
		\caption{Controlador Proporcional - Integral - Derivativo (kp = 10, ki = 15 e kd = 0.2)}
		\label{fig:contPIDP}
	\end{subfigure}
	\caption{Experimentos com Controlador PID}
	\label{fig:contPID}
\end{figure}


\section{Malha 1: Função de Ajuste Polinomial e Controlador Embutido}
\label{m1ContPol}
%Como já dito anteriormente, esse método utiliza de uma função polinomial que é um mapeamento aproximado da resposta dos motores aos diversos parâmetros de potência, tal como a \autoref{eq:pol}. Essa função foi obtida %pelo professor Tales Argolo 
%aplicando diversas potências ao motor e verificando a velocidade que o mesmo adquiria com esses comandos, feito isso foi utilizado o \emph{MATLAB\textregistered} para achar uma função aproximada de 4º grau. É importante constatar aqui que esse método utiliza um método da própria plataforma que implementa internamente um controlador, onde independente do peso contido no motor, ele adquiri a mesma velocidade para determinada potência(Função: \emph{OnFwdReg}).
Como já dito anteriormente, a plataforma já implementa uma malha de controle da velocidade dos motores, um controlador \emph{PID} já ajustado pelo fabricante. Para utilizá-la basta usar um método específico da linguagem \emph{NXC}. Entretanto, a referência desta malha de controle é uma variável que assume valores entre -100 e 100. Surge daí, portanto, a necessidade de se fazer um mapeamento entre esta variável e a variável de velocidade angular assumida pelo motor em radianos por segundo. Para realizar este mapeamento estático, utiliza-se o método dos mínimos quadrados para se ajustar um polinômio de quarto grau aos dados medidos, obtendo-se assim a seguinte expressão:% a função mostrada na \autoref{eq:pol}.
\begin{equation}
f(\omega_{desejada}) = -0.00091\omega_{desejada}^{4} + 0.0223\omega_{desejada}^{3} - 0.01537\omega_{desejada}^{2} + 6.1864\omega_{desejada} - 0.0546
\label{eq:pol}
\end{equation} 
 
 Como pode ser visto na \autoref{fig:pol}, a função mapeia muito bem as saídas do motor, o sistema converge para um valor de erro nulo com um tempo de aproximadamente 0,75 segundos, com uma oscilação de $\pm0.6 rad/s$, chegando a ser quase quatro vezes mais rápida que os resultados obtidos com o controlador PID ajustado na \autoref{m1ContPID}.  %bem próximo aos resultados obtidos com o controlador PI ajustado na \autoref{m1ContPID}. %manualmente. sendo, uma das mais rápidas, com menor oscilação e controle suave.
 
 \begin{figure}[!htb]
 	\centering
	\includegraphics[width=.9\linewidth]{./Testes/Malha1/Polinomial/Polinomial}
 	\caption{Experimentos com a Função Polinomial e o Controlador PID Embutido}
 	\label{fig:pol}
 \end{figure}
 
 \section{Malha 2: Controle de Posicionamento}
 \label{m2}
 Nesta parte do trabalho, serão apresentados os testes realizados com a segunda malha de controle, que é responsável pelo posicionamento do robô. Ou seja, a malha recebe como referência a posição onde deseja-se que o robô esteja, a partir daí calcula-se a velocidade angular necessária para que o robô chegue ao ponto desejado, e esta velocidade, por sua vez, é passada para a malha de controle mais interna. 
 
 Os testes foram realizados da seguinte maneira: O robô foi iniciado no ponto ($0.8,0.0$), com orientação ($\theta$) inicial igual a $\pi$, e foi definido um alvo na origem do sistema que esse robô teria que circundar com um raio de 30 centímetros e um período de 24 segundos. Após mil e duzentas amostras, o raio e o período do sistema mudam sendo 40 centímetros de raio em um período de 30 segundos, tornando o teste muito parecido com o segundo problema em si.
 
 Na \autoref{t:compMalha2} são apresentados alguns dos resultados obtidos com os experimentos feitos com a segunda malha, utilizando-se o controlador proporcional de ganho unitário e com ganho proporcional igual a 2 na malha intermediária, a malha responsável pelo controle de posicionamento. Como já sabido, quando tem-se um controle em cascata é necessário que a malha interna responda pelo menos 5 vezes mais rápido do que a malha externa, e quanto mais rápido melhor.
 
 Tendo isto em mente, %pode-se dizer que o a segunda malha de controle utilizando como controlador da malha interna o controlador polinomial, possui uma malha interna cerca de 7 vezes mais rápida do que a malha externa. Já com os controladores Incremental e PID são cerca de 14 e 11 vezes mais rápida, respectivamente.
 %Dessa forma, 
 do ponto de vista de tempo de resposta, %todas atendem com o pré-requisito
 todos os controladores da malha mais interna testados atendem ao pré-requisito, no entanto, ao observar o sistema na vida real é possível notar que o controlador PID é muito agressivo e por vezes produz um tranco no motor, o que não é desejado, visto que qualquer comando brusco pode ocasionar em derrapagem das rodas, o que poderia levar o robô a se perder por acumulo de erro odométrico e mapear uma trajetória diferente da trajetória apresentada no mundo real.
 
 Os dois que apresentaram melhores resultados foram o controlador da malha externa proporcional com ganho de aproximadamente igual a $2$ combinados com os controladores Incremental e Polinomial, o que se confirma pelas figuras \ref{fig:m2inc}, \ref{fig:m2pid}, \ref{fig:m2pol} e pela \autoref{t:compMalha2}. Como mostrado na \autoref{fig:m2pol2}, os testes com controlador PI na malha intermediária não apresentaram um bom resultado, apresentando oscilações indesejadas que demoraram para se estabilizar ao mudar a ação de controle drasticamente, além de apresentar um controle mais agressivo.
 
 \begin{table}[!htb]
 	\centering
 	\begin{tabular}{|l|l|l|l|}
 		\hline
 		\textbf{\begin{tabular}[c]{@{}l@{}}Características/\\ Controlador Interno e Externo\end{tabular}}          & \textbf{\begin{tabular}[c]{@{}l@{}}Estabiliza em\\ ($rad$)\end{tabular}} & \textbf{\begin{tabular}[c]{@{}l@{}}Termpo de \\ Assentamento ($s$)\end{tabular}} & \textbf{Controle}                                          \\ \hline
 		\textbf{\begin{tabular}[c]{@{}l@{}}Incremental (Incremento 3)/\\ Proporcional (kp = 1.0)\end{tabular}}     & $\pm 0.3rad$                                                                & $\pm 14 s$                                                                     & Suave                                                      \\ \hline
 		\textbf{\begin{tabular}[c]{@{}l@{}}PID (kp = 10;ki = 15;kd = 0.2)/\\ Proporcional (kp = 1.0)\end{tabular}} & $\pm0.3 rad$                                                                 & $\pm17.5 s$                                                                   & Agressivo                                                  \\ \hline
 		\textbf{\begin{tabular}[c]{@{}l@{}}Polinomial/\\ Proporcional (kp = 1.0)\end{tabular}}                     & $\pm 0.3rad$                                                                    & $\pm 17.5 s$                                                                   & Suave                                                      \\ \hline
 		\textbf{\begin{tabular}[c]{@{}l@{}}Incremental (Incremento 3)/\\ Proporcional (kp = 2.0)\end{tabular}}     & $\pm 0.18 rad$                                                                   & $\pm 11 s$                                                                     & \begin{tabular}[c]{@{}l@{}}Pouco \\ Agressivo\end{tabular} \\ \hline
 		\textbf{\begin{tabular}[c]{@{}l@{}}Polinomial/\\ Proporcional (kp = 2.0)\end{tabular}}                     & $\pm 0.17 rad$                                                                   & $\pm 14 s$                                                                     & Suave                                                      \\ \hline
 	\end{tabular}
 	\caption{Comparativo dos Controladores da Malha Intermediária (malha 2)}
 	\label{t:compMalha2}
 \end{table}
 
 \begin{figure}[!htb]
 	\centering
 	\begin{subfigure}{1.0\textwidth}
 		\centering
 		\includegraphics[width=.9\linewidth]{./Testes/Malha2/P2.0/Incremental-3/m2PosInc}
 		\caption{Malha Intermediária: P=2.0 - Malha Interna: Incremental k = 3}
 		\label{fig:m2incpos}
 	\end{subfigure}
 	\begin{subfigure}{1.0\textwidth}
 		\centering
 		\includegraphics[width=.9\linewidth]{./Testes/Malha2/P2.0/Incremental-3/ErroM2Inc}
 		\caption{Erro da Malha Intermediária: P=2.0 - Malha Interna: Incremental k = 3}
 		\label{fig:m2incerr}
 	\end{subfigure}
 	\caption{Experimentos com a Malha Intermediária com Controlador Proporcional(kp=2.0) e Malha Interna: Controlador Incremental (k = 3)}
 	\label{fig:m2inc}
 \end{figure}
 
  \begin{figure}[!htb]
  	\centering
  	\begin{subfigure}{1.0\textwidth}
  		\centering
  		\includegraphics[width=.9\linewidth]{./Testes/Malha2/P1.0/PID/PIDPos}
  		\caption{Malha Intermediária: P=1.0 - Malha Interna: PID (kp = 10;ki = 15; kd = 0.2)}
  		\label{fig:m2pidpos}
  	\end{subfigure}
  	\begin{subfigure}{1.0\textwidth}
  		\centering
  		\includegraphics[width=.9\linewidth]{./Testes/Malha2/P1.0/PID/ErroThetaPID}
  		\caption{Erro da Malha Intermediária: P=1.0 - Malha Interna: PID (kp = 10;ki = 15; kd = 0.2)}
  		\label{fig:m2pid2pos}
  	\end{subfigure}
  	\caption{Experimentos com a Malha Intermediária com Controlador Proporcional e Malha Interna: Controlador PID}
  	\label{fig:m2pid}
  \end{figure}
  
    \begin{figure}[!htb]
    	\centering
    	\begin{subfigure}{1.0\textwidth}
    		\centering
    		\includegraphics[width=.9\linewidth]{./Testes/Malha2/P2.0/Polinomial/m2PosPol}
    		\caption{Malha Intermediária: P=2.0 - Malha Interna: Polinomial}
    		\label{fig:m2polpos}
    	\end{subfigure}
    	\begin{subfigure}{1.0\textwidth}
    		\centering
    		\includegraphics[width=.9\linewidth]{./Testes/Malha2/P2.0/Polinomial/m2ErroTheta}
    		\caption{Erro da Malha Intermediária: P=2.0 - Malha Interna: Polinomial}
    		\label{fig:m2pol2pos}
    	\end{subfigure}
    	\caption{Experimentos com a Malha Intermediária com Controlador Proporcional e Malha Interna: Controlador Polinomial}
    	\label{fig:m2pol}
    \end{figure}
 
    \begin{figure}[!htb]
    	\centering
    	\begin{subfigure}{1.0\textwidth}
    		\centering
    		\includegraphics[width=.9\linewidth]{./Testes/Malha2/P1.0I0.3/IntegralOsc}
    		\caption{Malha Intermediária: P=1.0; I=0.3 - Malha Interna: Polinomial}
    		\label{fig:m2polpos2}
    	\end{subfigure}
    	\begin{subfigure}{1.0\textwidth}
    		\centering
    		\includegraphics[width=.9\linewidth]{./Testes/Malha2/P1.0I0.3/ErroThetaIOsc}
    		\caption{Erro da Malha Intermediária: P=1.0; I=0.3 - Malha Interna: Polinomial}
    		\label{fig:m2piderr2}
    	\end{subfigure}
    	\caption{Experimentos com a Malha Intermediária com Controlador Proporcional e Malha Interna: Controlador Polinomial}
    	\label{fig:m2pol2}
    \end{figure}



