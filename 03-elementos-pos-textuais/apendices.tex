%
% Documento: Apêndices
%

\begin{apendicesenv}
\partapendices

\chapter{Código do problema 1: Mestre}
\label{chap:codP1Mestre}

\begin{lstlisting}
//Mestre P1 - Mestre
#define BT_CONN 1
#define INBOX 1
#define OUTBOX 5

#define L 0.112
#define RP 0.0216
#define motorR OUT_A
#define motorL OUT_C
float R = 0.3;
float T = 24;

//Auxiliar para gravar no arquivo txt
string write;
short bytesWritten;
byte fileHandle;
string write2;
short bytesWritten2;
byte fileHandle2;

//Converter graus em radianos 1grau = 0,0174533 rad
float convrad = 0.0174533;

int t_amostral = 25;
int amostras = 3000;

long tempo_k = 0.0;
float dt = 0.0;

float countR_k = 0;
float countL_k = 0;
float countR_kplus = 0;
float countL_kplus = 0;

float countR2_k = 0;
float countL2_k = 0;
float countR2_kplus = 0;
float countL2_kplus = 0;

float wr = 0.0;
float wl = 0.0;
float wdr = 0.0;
float wdl = 0.0;
float ewr = 0.0;
float ewl = 0.0;

float errx = 0.0;
float erry = 0.0;
float errt = 0.0;

//Variaveis do Controlador
float acum_err = 0.0;
float acum_erl = 0.0;
float prev_ewr = 0.0;
float prev_ewl = 0.0;
//Ganhos
float kp = 10;
float ki = 15;
float kd = 0;
float incremento = 3;

//Ganhos
float kp_pos = 2.0;
float ki_pos = 0;
float kd_pos = 0;
float incremento_pos = 1;

float acum_errt = 0;
float errt_past = 0;
float w_controle = 0;

//Posicionamento Inicial do Robo
float xr = 0.0;
float yr = 0.0;
float tetar_k = PI;
float tetad = PI;

float cr = 0;
float cl = 0;

float ddr = 0;
float ddl = 0;

float P[] = {-0.00091, 0.0223, -0.1537, 6.1864, -0.0546};

string out;
string in;

float vd = 0.1;

sub getWs(){
countR_kplus = MotorRotationCount(motorR);
countL_kplus = MotorRotationCount(motorL);

//Velocidade = delta_Rotacao*constConvGrau2Rad/delta_tempo em segundos
wr = (countR_kplus - countR_k)*convrad/dt;  //wr em rad/s
wl = (countL_kplus - countL_k)*convrad/dt;  //wl em rad/s
countR_k = countR_kplus;
countL_k = countL_kplus;

}

sub malha1(float vLinear,float vAng){

//Definindo as velocidades desejadas de Cada Roda
wdr = (2*vLinear + vAng*L)/(2*RP);
wdl = (2*vLinear - vAng*L)/(2*RP);

//Obtendo as velocidades reais de cada roda
getWs();

//Obtendo os erros de velocidade
ewr = wdr - wr;
ewl = wdl - wl;

//Acoes de controle
//Controle Proporcional
/*cr = kp*ewr;
cl = kp*ewl;*/
//Controle Proporcional Integrador
/*cr = kp*ewr + acum_err + ki*dt*ewr;
cl = kp*ewl + acum_erl + ki*dt*ewl;*/
//Controlador Proporcional Integrador Derivativo
/*cr = kp*ewr + acum_err + ki*dt*ewr + kd*(ewr - prev_ewr)/dt;
cl = kp*ewl + acum_erl + ki*dt*ewl + kd*(ewl - prev_ewl)/dt;
//Controlador Intedral e Derivativo
//Salvando erro atual
prev_ewr = ewr;
prev_ewl = ewl;

//Atualizando o acumulador de erro
acum_err = acum_err + ki*dt*ewr;
acum_erl = acum_erl + ki*dt*ewl;   */

//Incremental
/*if(ewr > 0){
cr = cr + incremento;
}else{
if(ewr < 0 ){
cr = cr - incremento;
}
}
if(ewl > 0){
cl = cl + incremento;
}else{
if(ewr < 0 ){
cl = cl - incremento;
}
}*/

/*
if(cr > 100){
cr = 100;
}
if(cr < -100){
cr = -100;
}
if(cl > 100){
cl = 100;
}
if(cl < -100){
cl = -100;
}

OnFwd(OUT_A,cr);
OnFwd(OUT_C,cl);    */


//Variaveis de Ajuste Polinomial
cr = P[0]*pow(wdr,4) + P[1]*pow(wdr,3) + P[2]*pow(wdr,2) + P[3]*wdr + P[4];  // Ajuste polinomial
cl = P[0]*pow(wdl,4) + P[1]*pow(wdl,3) + P[2]*pow(wdl,2) + P[3]*wdl + P[4];  // Pwr_{r,l} X w_{r,l}
// Comandos para os motores (Polinomial)
OnFwdReg(motorR, cr, OUT_REGMODE_SPEED);
OnFwdReg(motorL, cl, OUT_REGMODE_SPEED);


//Salvando erros em arquivo
write = StrCat(NumToStr(ewr),"|",NumToStr(ewl),"|",NumToStr(dt));
WriteLnString(fileHandle,write, bytesWritten);
}

sub malha2(/*float xd, float yd*/){
//Na verdade so atualiza a posicao real dos robos (utilizada para primeira abordagem)
//Obtendo distancias percorridas por cada roda
countR2_kplus = MotorRotationCount(motorR);
countL2_kplus = MotorRotationCount(motorL);

ddr = ((countR2_kplus - countR2_k) * 2 * PI * RP)/360.0;
ddl = ((countL2_kplus - countL2_k) * 2 * PI * RP)/360.0;
countR2_k = countR2_kplus;
countL2_k = countL2_kplus;

//Obtendo posicao e sentido real
xr = xr + ((ddr+ddl)* cos(tetar_k)/2.0);
yr = yr + ((ddr+ddl)* sin(tetar_k)/2.0);
tetar_k = tetar_k + (ddr - ddl)/L;

//Salvando erros em arquivo
short bytesWritten2;
string write2 = StrCat(NumToStr(xr),"|",NumToStr(yr));
WriteLnString(fileHandle2,write2, bytesWritten2);

}

sub BTCheck(int conn){
if (!BluetoothStatus(conn)==NO_ERR){
TextOut(5,LCD_LINE2,"Error");
Wait(1000);
Stop(true);
}
}

task main(){

DeleteFile("Tst_Malha1.txt");
CreateFile("Tst_Malha1.txt",8*2048, fileHandle);
DeleteFile("Tst_Malha2.txt");
CreateFile("Tst_Malha2.txt",30*2048, fileHandle2);
//write = StrCat("kp: ",NumToStr(kp),"|ki: ",NumToStr(ki),"|kd: ",NumToStr(kd));
//WriteLnString(fileHandle,write, bytesWritten);

BTCheck(BT_CONN); //checa a conexao com o master
float x1;
tempo_k = CurrentTick();
float tempo_inicial = tempo_k;
for(int t = 0; t < amostras; t++){
malha2();
/*******************/
out = StrCat("ok|", NumToStr(xr));
TextOut(10,LCD_LINE1,"Master Test");
TextOut(0,LCD_LINE2,"IN:");
TextOut(0,LCD_LINE3,"OUT:");
ReceiveRemoteString(INBOX, true, in);
SendRemoteString(BT_CONN,OUTBOX,out);
TextOut(10,LCD_LINE3,out);

int index = Pos("|",in);
if(strcmp(Copy(in,0,index), "ok") == 0){
//vi = 0.25;
x1 = atof(Copy(in,index+1,StrLen(in)-index+1));
TextOut(10,LCD_LINE2,in);

}
Wait(100);
//Calculando dt em segundos
long dt_aux = CurrentTick() - tempo_k;
tempo_k = tempo_k + dt_aux;
dt = dt_aux/1000.0;

//positionControl();
Wait(t_amostral);
/*********/

/*float xd = (xr+x1)/2;
float yd = 0.0;
malha2(xd,yd);*/
float v = vd + 0.2*(xr - x1);
malha1(v,0);

Wait(t_amostral);
}

CloseFile(fileHandle);

}
\end{lstlisting}

\chapter{Código do problema 1: Escravo}
\label{chap:codP1Escravo}

\begin{lstlisting}
//Mestre P1 - Escravo
#define BT_CONN 0
#define INBOX 5
#define OUTBOX 1

#define L 0.112
#define RP 0.0216
#define motorR OUT_A
#define motorL OUT_C
float R = 0.3;
float T = 24;

//Auxiliar para gravar no arquivo txt
string write;
short bytesWritten;
byte fileHandle;
string write2;
short bytesWritten2;
byte fileHandle2;

//Converter graus em radianos 1grau = 0,0174533 rad
float convrad = 0.0174533;

int t_amostral = 25;
int amostras = 3000;

long tempo_k = 0.0;
float dt = 0.0;

float countR_k = 0;
float countL_k = 0;
float countR_kplus = 0;
float countL_kplus = 0;

float countR2_k = 0;
float countL2_k = 0;
float countR2_kplus = 0;
float countL2_kplus = 0;

float wr = 0.0;
float wl = 0.0;
float wdr = 0.0;
float wdl = 0.0;
float ewr = 0.0;
float ewl = 0.0;

float errx = 0.0;
float erry = 0.0;
float errt = 0.0;

//Variaveis do Controlador
float acum_err = 0.0;
float acum_erl = 0.0;
float prev_ewr = 0.0;
float prev_ewl = 0.0;
//Ganhos
float kp = 10;
float ki = 15;
float kd = 0;
float incremento = 3;

//Ganhos
float kp_pos = 2.0;
float ki_pos = 0;
float kd_pos = 0;
float incremento_pos = 1;

float acum_errt = 0;
float errt_past = 0;
float w_controle = 0;

//Posicionamento Inicial do Robo
float xr = 0.8;
float yr = 0.0;
float tetar_k = PI;
float tetad = PI;

float cr = 0;
float cl = 0;

float ddr = 0;
float ddl = 0;

float P[] = {-0.00091, 0.0223, -0.1537, 6.1864, -0.0546};

string out;
string in;

float vd = 0.1;

sub getWs(){
countR_kplus = MotorRotationCount(motorR);
countL_kplus = MotorRotationCount(motorL);

//Velocidade = delta_Rotacao*constConvGrau2Rad/delta_tempo em segundos
wr = (countR_kplus - countR_k)*convrad/dt;  //wr em rad/s
wl = (countL_kplus - countL_k)*convrad/dt;  //wl em rad/s
countR_k = countR_kplus;
countL_k = countL_kplus;

}

sub malha1(float vLinear,float vAng){

//Definindo as velocidades desejadas de Cada Roda
wdr = (2*vLinear + vAng*L)/(2*RP);
wdl = (2*vLinear - vAng*L)/(2*RP);

//Obtendo as velocidades reais de cada roda
getWs();

//Obtendo os erros de velocidade
ewr = wdr - wr;
ewl = wdl - wl;

//Acoes de controle
//Controle Proporcional
/*cr = kp*ewr;
cl = kp*ewl;*/
//Controle Proporcional Integrador
/*cr = kp*ewr + acum_err + ki*dt*ewr;
cl = kp*ewl + acum_erl + ki*dt*ewl;*/
//Controlador Proporcional Integrador Derivativo
/*cr = kp*ewr + acum_err + ki*dt*ewr + kd*(ewr - prev_ewr)/dt;
cl = kp*ewl + acum_erl + ki*dt*ewl + kd*(ewl - prev_ewl)/dt;
//Controlador Intedral e Derivativo
//Salvando erro atual
prev_ewr = ewr;
prev_ewl = ewl;

//Atualizando o acumulador de erro
acum_err = acum_err + ki*dt*ewr;
acum_erl = acum_erl + ki*dt*ewl;   */

//Incremental
/*if(ewr > 0){
cr = cr + incremento;
}else{
if(ewr < 0 ){
cr = cr - incremento;
}
}
if(ewl > 0){
cl = cl + incremento;
}else{
if(ewr < 0 ){
cl = cl - incremento;
}
}*/

/*
if(cr > 100){
cr = 100;
}
if(cr < -100){
cr = -100;
}
if(cl > 100){
cl = 100;
}
if(cl < -100){
cl = -100;
}

OnFwd(OUT_A,cr);
OnFwd(OUT_C,cl);    */


//Variaveis de Ajuste Polinomial
cr = P[0]*pow(wdr,4) + P[1]*pow(wdr,3) + P[2]*pow(wdr,2) + P[3]*wdr + P[4];  // Ajuste polinomial
cl = P[0]*pow(wdl,4) + P[1]*pow(wdl,3) + P[2]*pow(wdl,2) + P[3]*wdl + P[4];  // Pwr_{r,l} X w_{r,l}
// Comandos para os motores (Polinomial)
OnFwdReg(motorR, cr, OUT_REGMODE_SPEED);
OnFwdReg(motorL, cl, OUT_REGMODE_SPEED);


//Salvando erros em arquivo
write = StrCat(NumToStr(ewr),"|",NumToStr(ewl),"|",NumToStr(dt));
WriteLnString(fileHandle,write, bytesWritten);
}

sub malha2(/*float xd, float yd*/){
//Nao implementa malha 2 e sim uma adaptacao. Apenas atualiza a posicao real
//Obtendo distancias percorridas por cada roda
countR2_kplus = MotorRotationCount(motorR);
countL2_kplus = MotorRotationCount(motorL);

ddr = ((countR2_kplus - countR2_k) * 2 * PI * RP)/360.0;
ddl = ((countL2_kplus - countL2_k) * 2 * PI * RP)/360.0;
countR2_k = countR2_kplus;
countL2_k = countL2_kplus;

//Obtendo posicao e sentido real
xr = xr + ((ddr+ddl)* cos(tetar_k)/2.0);
yr = yr + ((ddr+ddl)* sin(tetar_k)/2.0);
tetar_k = tetar_k + (ddr - ddl)/L;

//Salvando erros em arquivo
short bytesWritten2;
string write2 = StrCat(NumToStr(xr),"|",NumToStr(yr));
WriteLnString(fileHandle2,write2, bytesWritten2);

}

sub BTCheck(int conn){
if (!BluetoothStatus(conn)==NO_ERR){
TextOut(5,LCD_LINE2,"Error");
Wait(1000);
Stop(true);
}
}

task main(){

DeleteFile("Tst_Malha1.txt");
CreateFile("Tst_Malha1.txt",8*2048, fileHandle);
DeleteFile("Tst_Malha2.txt");
CreateFile("Tst_Malha2.txt",30*2048, fileHandle2);
//write = StrCat("kp: ",NumToStr(kp),"|ki: ",NumToStr(ki),"|kd: ",NumToStr(kd));
//WriteLnString(fileHandle,write, bytesWritten);

BTCheck(BT_CONN); //checa a conexao com o master
float x1;
tempo_k = CurrentTick();
float tempo_inicial = tempo_k;
for(int t = 0; t < amostras; t++){
malha2();
/*******************/
out = StrCat("ok|", NumToStr(xr));
TextOut(10,LCD_LINE1,"Master Test");
TextOut(0,LCD_LINE2,"IN:");
TextOut(0,LCD_LINE3,"OUT:");
ReceiveRemoteString(INBOX, true, in);
SendResponseString(OUTBOX,out);
TextOut(10,LCD_LINE3,out);

int index = Pos("|",in);
if(strcmp(Copy(in,0,index), "ok") == 0){
//vi = 0.25;
x1 = atof(Copy(in,index+1,StrLen(in)-index+1));
TextOut(10,LCD_LINE2,in);

}
Wait(100);
//Calculando dt em segundos
long dt_aux = CurrentTick() - tempo_k;
tempo_k = tempo_k + dt_aux;
dt = dt_aux/1000.0;

//positionControl();
Wait(t_amostral);
/*********/

/*float xd = (xr+x1)/2;
float yd = 0.0;
malha2(xd,yd);*/
float v = vd + 0.2*(xr - x1);
TextOut(0,LCD_LINE4,NumToStr(v));
malha1(v,0);

Wait(t_amostral);
}

CloseFile(fileHandle);

}
\end{lstlisting}

\chapter{Código do problema 2: Mestre}
\label{chap:codP2Mestre}

\begin{lstlisting}
//Mestre P2 - Mestre
#define BT_CONN 1
#define INBOX 1
#define OUTBOX 5

#define L 0.112
#define RP 0.0216
#define motorR OUT_A
#define motorL OUT_C
float R = 0.3;
float T = 12;

//Auxiliar para gravar no arquivo txt
string write;
short bytesWritten;
byte fileHandle;
string write2;
short bytesWritten2;
byte fileHandle2;

//Converter graus em radianos 1grau = 0,0174533 rad
float convrad = 0.0174533;

int t_amostral = 25;
int amostras = 3000;

long tempo_k = 0.0;
float dt = 0.0;

float countR_k = 0;
float countL_k = 0;
float countR_kplus = 0;
float countL_kplus = 0;

float countR2_k = 0;
float countL2_k = 0;
float countR2_kplus = 0;
float countL2_kplus = 0;

float wr = 0.0;
float wl = 0.0;
float wdr = 0.0;
float wdl = 0.0;
float ewr = 0.0;
float ewl = 0.0;

float errx = 0.0;
float erry = 0.0;
float errt = 0.0;

//Variaveis do Controlador
float acum_err = 0.0;
float acum_erl = 0.0;
float prev_ewr = 0.0;
float prev_ewl = 0.0;
//Ganhos
float kp = 10;
float ki = 15;
float kd = 0;
float incremento = 3;

//Ganhos
float kp_pos = 2.0;
float ki_pos = 0;
float kd_pos = 0;
float incremento_pos = 1;

float acum_errt = 0;
float errt_past = 0;
float w_controle = 0;

//Posicionamento Inicial do Robo
float xr = 0.8;
float yr = 0.0;
float tetar_k = PI;
float tetad = PI;

float cr = 0;
float cl = 0;

float ddr = 0;
float ddl = 0;

float P[] = {-0.00091, 0.0223, -0.1537, 6.1864, -0.0546};

string out;
string in;

float vd = 0.1;

int num_robos = 1;
int num = 0;
float wt = 0;

sub getWs(){
countR_kplus = MotorRotationCount(motorR);
countL_kplus = MotorRotationCount(motorL);

//Velocidade = delta_Rotacao*constConvGrau2Rad/delta_tempo em segundos
wr = (countR_kplus - countR_k)*convrad/dt;  //wr em rad/s
wl = (countL_kplus - countL_k)*convrad/dt;  //wl em rad/s
countR_k = countR_kplus;
countL_k = countL_kplus;

}

sub malha1(float vLinear,float vAng){

//Definindo as velocidades desejadas de Cada Roda
wdr = (2*vLinear + vAng*L)/(2*RP);
wdl = (2*vLinear - vAng*L)/(2*RP);

//Obtendo as velocidades reais de cada roda
getWs();

//Obtendo os erros de velocidade
ewr = wdr - wr;
ewl = wdl - wl;

//Acoes de controle
//Controle Proporcional
/*cr = kp*ewr;
cl = kp*ewl;*/
//Controle Proporcional Integrador
/*cr = kp*ewr + acum_err + ki*dt*ewr;
cl = kp*ewl + acum_erl + ki*dt*ewl;*/
//Controlador Proporcional Integrador Derivativo
/*cr = kp*ewr + acum_err + ki*dt*ewr + kd*(ewr - prev_ewr)/dt;
cl = kp*ewl + acum_erl + ki*dt*ewl + kd*(ewl - prev_ewl)/dt;
//Controlador Intedral e Derivativo
//Salvando erro atual
prev_ewr = ewr;
prev_ewl = ewl;

//Atualizando o acumulador de erro
acum_err = acum_err + ki*dt*ewr;
acum_erl = acum_erl + ki*dt*ewl;   */

//Incremental
/*if(ewr > 0){
cr = cr + incremento;
}else{
if(ewr < 0 ){
cr = cr - incremento;}
}
if(ewl > 0){
cl = cl + incremento;
}else{
if(ewr < 0 ){
cl = cl - incremento;}
}

if(cr > 100){
cr = 100;
}
if(cr < -100){
cr = -100;
}
if(cl > 100){
cl = 100;
}
if(cl < -100){
cl = -100;
}

OnFwd(OUT_A,cr);
OnFwd(OUT_C,cl);*/


//Variaveis de Ajuste Polinomial
cr = P[0]*pow(wdr,4) + P[1]*pow(wdr,3) + P[2]*pow(wdr,2) + P[3]*wdr + P[4];  // Ajuste polinomial
cl = P[0]*pow(wdl,4) + P[1]*pow(wdl,3) + P[2]*pow(wdl,2) + P[3]*wdl + P[4];  // Pwr_{r,l} X w_{r,l}
// Comandos para os motores (Polinomial)
OnFwdReg(motorR, cr, OUT_REGMODE_SPEED);
OnFwdReg(motorL, cl, OUT_REGMODE_SPEED);   


//Salvando erros em arquivo
write = StrCat(NumToStr(ewr),"|",NumToStr(ewl),"|",NumToStr(dt));
WriteLnString(fileHandle,write, bytesWritten);
}

sub malha2(float xd, float yd){
//Obtendo distancias percorridas por cada roda
countR2_kplus = MotorRotationCount(motorR);
countL2_kplus = MotorRotationCount(motorL);

ddr = ((countR2_kplus - countR2_k) * 2 * PI * RP)/360.0;
ddl = ((countL2_kplus - countL2_k) * 2 * PI * RP)/360.0;
countR2_k = countR2_kplus;
countL2_k = countL2_kplus;

//Obtendo posicao e sentido real
xr = xr + ((ddr+ddl)* cos(tetar_k)/2.0);
yr = yr + ((ddr+ddl)* sin(tetar_k)/2.0);
tetar_k = tetar_k + (ddr - ddl)/L;

errx = xd - xr;
erry = yd - yr;

tetad = atan2(erry,errx);

errt_past = errt;
//Diminuir de acordo com o erro
//velLinear_Desejada = 2 * sqrt((pow(errx,2)+pow(erry,2)));

errt = tetad - tetar_k;
errt = atan2(sin(errt),cos(errt));

//Controle P
//float w_controled = kp_pos*errt;

//Controle PI
//w_controled = kp_pos*errt + acum_errt + ki_pos*dt*errt;
//Controle PID
//float w_controled = kp_pos*errt + acum_errt + ki_pos*dt*errt + kd_pos * (errt - errt_past)/dt;
w_controle = kp_pos*errt + acum_errt + ki_pos*errt*dt + kd_pos * (errt - errt_past)/dt;
acum_errt = acum_errt + ki_pos*errt*dt;
acum_errt = acum_errt + ki_pos*dt*errt;

//Salvando erros em arquivo
short bytesWritten2;
string write2 = StrCat(NumToStr(w_controle));
WriteLnString(fileHandle2,write2, bytesWritten2);

float dist = sqrt((pow(errx,2)+(pow(erry,2))));
float v = (2*PI*R/T)/num_robos;
if(dist > 0.05){
v = v*1.1;
}
malha1(v, w_controle);
}


sub BTCheck(int conn){
if (!BluetoothStatus(conn)==NO_ERR){
TextOut(5,LCD_LINE2,"Error");
Wait(1000);
Stop(true);
}
}

task main(){

DeleteFile("Mestre_2.txt");
CreateFile("Mestre_2.txt",38*2048, fileHandle2);
DeleteFile("Mestre_1.txt");
//CreateFile("Mestre_1.txt",8*2048, fileHandle);
//write = StrCat("kp: ",NumToStr(kp),"|ki: ",NumToStr(ki),"|kd: ",NumToStr(kd));
//WriteLnString(fileHandle,write, bytesWritten);

BTCheck(BT_CONN); //checa a conexao com o master
float x1;
tempo_k = CurrentTick();
float tempo_inicial = tempo_k;
for(int t = 0; t < amostras; t++){
/*******************/
if(!BluetoothStatus(BT_CONN)==NO_ERR){
num_robos = 1;
}else{
num_robos = 2;
}
/*********************/
//Calculando dt em segundos
long dt_aux = CurrentTick() - tempo_k;
tempo_k = tempo_k + dt_aux;
dt = dt_aux/1000.0;

/*********************/

wt = ((tempo_k/1000.0)*(2*PI/T))/num_robos;

/*********************/

out = StrCat(NumToStr(num_robos),"|", NumToStr(wt));
TextOut(10,LCD_LINE1,"Master Test");
//        TextOut(0,LCD_LINE2,"IN:");
TextOut(0,LCD_LINE3,"OUT:");
//ReceiveRemoteString(INBOX, true, in);
SendRemoteString(BT_CONN,OUTBOX,out);
TextOut(10,LCD_LINE3,out);

float xd = R*cos(wt + num*(2*PI)/num_robos);
float yd = R*sin(wt + num*(2*PI)/num_robos);
malha2(xd,yd);
Wait(200);
//Wait(t_amostral);
}

CloseFile(fileHandle);

}
\end{lstlisting}


\chapter{Código do problema 2: Escravo}
\label{chap:codP2Escravo}

\begin{lstlisting}
//Mestre P2 - Mestre
#define BT_CONN 0
#define INBOX 5
#define OUTBOX 1

#define L 0.112
#define RP 0.0216
#define motorR OUT_A
#define motorL OUT_C
float R = 0.3;
float T = 12;

//Auxiliar para gravar no arquivo txt
string write;
short bytesWritten;
byte fileHandle;
string write2;
short bytesWritten2;
byte fileHandle2;

//Converter graus em radianos 1grau = 0,0174533 rad
float convrad = 0.0174533;

int t_amostral = 25;
int amostras = 3000;

long tempo_k = 0.0;
float dt = 0.0;

float countR_k = 0;
float countL_k = 0;
float countR_kplus = 0;
float countL_kplus = 0;

float countR2_k = 0;
float countL2_k = 0;
float countR2_kplus = 0;
float countL2_kplus = 0;

float wr = 0.0;
float wl = 0.0;
float wdr = 0.0;
float wdl = 0.0;
float ewr = 0.0;
float ewl = 0.0;

float errx = 0.0;
float erry = 0.0;
float errt = 0.0;

//Variaveis do Controlador
float acum_err = 0.0;
float acum_erl = 0.0;
float prev_ewr = 0.0;
float prev_ewl = 0.0;
//Ganhos
float kp = 10;
float ki = 15;
float kd = 0;
float incremento = 3;

//Ganhos
float kp_pos = 2.0;
float ki_pos = 0;
float kd_pos = 0;
float incremento_pos = 1;

float acum_errt = 0;
float errt_past = 0;
float w_controle = 0;

//Posicionamento Inicial do Robo
float xr = -0.8;
float yr = 0.0;
float tetar_k = 0;
float tetad = 0;

float cr = 0;
float cl = 0;

float ddr = 0;
float ddl = 0;

float P[] = {-0.00091, 0.0223, -0.1537, 6.1864, -0.0546};

string out;
string in;

float vd = 0.1;

int num_robos = 1;
int num = 1;
float wt = 0;

sub getWs(){
countR_kplus = MotorRotationCount(motorR);
countL_kplus = MotorRotationCount(motorL);

//Velocidade = delta_Rotacao*constConvGrau2Rad/delta_tempo em segundos
wr = (countR_kplus - countR_k)*convrad/dt;  //wr em rad/s
wl = (countL_kplus - countL_k)*convrad/dt;  //wl em rad/s
countR_k = countR_kplus;
countL_k = countL_kplus;

}

sub malha1(float vLinear,float vAng){

//Definindo as velocidades desejadas de Cada Roda
wdr = (2*vLinear + vAng*L)/(2*RP);
wdl = (2*vLinear - vAng*L)/(2*RP);

//Obtendo as velocidades reais de cada roda
getWs();

//Obtendo os erros de velocidade
ewr = wdr - wr;
ewl = wdl - wl;

//Acoes de controle
//Controle Proporcional
/*cr = kp*ewr;
cl = kp*ewl;*/
//Controle Proporcional Integrador
/*cr = kp*ewr + acum_err + ki*dt*ewr;
cl = kp*ewl + acum_erl + ki*dt*ewl;*/
//Controlador Proporcional Integrador Derivativo
/*cr = kp*ewr + acum_err + ki*dt*ewr + kd*(ewr - prev_ewr)/dt;
cl = kp*ewl + acum_erl + ki*dt*ewl + kd*(ewl - prev_ewl)/dt;
//Controlador Intedral e Derivativo
//Salvando erro atual
prev_ewr = ewr;
prev_ewl = ewl;

//Atualizando o acumulador de erro
acum_err = acum_err + ki*dt*ewr;
acum_erl = acum_erl + ki*dt*ewl;   */

//Incremental
/*if(ewr > 0){
cr = cr + incremento;
}else{
if(ewr < 0 ){
cr = cr - incremento;}
}
if(ewl > 0){
cl = cl + incremento;
}else{
if(ewr < 0 ){
cl = cl - incremento;}
}


if(cr > 100){
cr = 100;}
if(cr < -100){
cr = -100;}
if(cl > 100){
cl = 100;}
if(cl < -100){
cl = -100;}

OnFwd(OUT_A,cr);
OnFwd(OUT_C,cl);*/


//Variaveis de Ajuste Polinomial
cr = P[0]*pow(wdr,4) + P[1]*pow(wdr,3) + P[2]*pow(wdr,2) + P[3]*wdr + P[4];  // Ajuste polinomial
cl = P[0]*pow(wdl,4) + P[1]*pow(wdl,3) + P[2]*pow(wdl,2) + P[3]*wdl + P[4];  // Pwr_{r,l} X w_{r,l}
// Comandos para os motores (Polinomial)
OnFwdReg(motorR, cr, OUT_REGMODE_SPEED);
OnFwdReg(motorL, cl, OUT_REGMODE_SPEED);


//Salvando erros em arquivo
write = StrCat(NumToStr(ewr),"|",NumToStr(ewl),"|",NumToStr(dt));
WriteLnString(fileHandle,write, bytesWritten);
}

sub malha2(float xd, float yd){
//Obtendo distancias percorridas por cada roda
countR2_kplus = MotorRotationCount(motorR);
countL2_kplus = MotorRotationCount(motorL);

ddr = ((countR2_kplus - countR2_k) * 2 * PI * RP)/360.0;
ddl = ((countL2_kplus - countL2_k) * 2 * PI * RP)/360.0;
countR2_k = countR2_kplus;
countL2_k = countL2_kplus;

//Obtendo posicao e sentido real
xr = xr + ((ddr+ddl)* cos(tetar_k)/2.0);
yr = yr + ((ddr+ddl)* sin(tetar_k)/2.0);
tetar_k = tetar_k + (ddr - ddl)/L;

errx = xd - xr;
erry = yd - yr;

tetad = atan2(erry,errx);

errt_past = errt;
//Diminuir de acordo com o erro
//velLinear_Desejada = 2 * sqrt((pow(errx,2)+pow(erry,2)));

errt = tetad - tetar_k;
errt = atan2(sin(errt),cos(errt));

//Controle P
//float w_controled = kp_pos*errt;

//Controle PI
//w_controled = kp_pos*errt + acum_errt + ki_pos*dt*errt;
//Controle PID
//float w_controled = kp_pos*errt + acum_errt + ki_pos*dt*errt + kd_pos * (errt - errt_past)/dt;
w_controle = kp_pos*errt + acum_errt + ki_pos*errt*dt + kd_pos * (errt - errt_past)/dt;
acum_errt = acum_errt + ki_pos*errt*dt;
acum_errt = acum_errt + ki_pos*dt*errt;

//Salvando erros em arquivo
short bytesWritten2;
string write2 = StrCat(NumToStr(xr),"|",NumToStr(yr));
WriteLnString(fileHandle2,write2, bytesWritten2);

float dist = sqrt((pow(errx,2)+(pow(erry,2))));
float v = (2*PI*R/T)/num_robos;
if(dist > 0.05){
v = v*1.1;
}
//v = v + dist*0.2;
/*if(dist < 0.05){
v = 0;
w_controle = 0;
} */
malha1(v, w_controle);
}


sub BTCheck(int conn){
if (!BluetoothStatus(conn)==NO_ERR){
TextOut(5,LCD_LINE2,"Error");
Wait(1000);
Stop(true);}
}

task main(){

DeleteFile("Escravo_2.txt");
CreateFile("Escravo_2.txt",38*2048, fileHandle2);
DeleteFile("Escravo_1.txt");
//CreateFile("Escravo_1.txt",8*2048, fileHandle);
//write = StrCat("kp: ",NumToStr(kp),"|ki: ",NumToStr(ki),"|kd: ",NumToStr(kd));
//WriteLnString(fileHandle,write, bytesWritten);

BTCheck(BT_CONN); //checa a conexao com o master
float x1;
tempo_k = CurrentTick();
float tempo_inicial = tempo_k;
for(int t = 0; t < amostras; t++){
/*******************/
if(!BluetoothStatus(BT_CONN)==NO_ERR){
num_robos = 1;
}else{
num_robos = 2;
}
/*********************/
//Calculando dt em segundos
long dt_aux = CurrentTick() - tempo_k;
tempo_k = tempo_k + dt_aux;
dt = dt_aux/1000.0;

TextOut(10,LCD_LINE1,"Master Test");
TextOut(0,LCD_LINE2,"IN:");
TextOut(0,LCD_LINE3,"OUT:");
ReceiveRemoteString(INBOX, true, in);

int index = Pos("|",in);
num_robos = StrToNum(Copy(in,0,index));
wt = atof(Copy(in,index+1,StrLen(in)-index+1));
TextOut(10,LCD_LINE2,in);

float xd = R*cos(wt + num*(2*PI)/num_robos);
float yd = R*sin(wt + num*(2*PI)/num_robos);
malha2(xd,yd);
Wait(200);
//Wait(t_amostral);
}

CloseFile(fileHandle);
CloseFile(fileHandle2);
}
\end{lstlisting}
\end{apendicesenv}
